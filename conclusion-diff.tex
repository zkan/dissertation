\setlength{\footskip}{8mm}

\chapter{Conclusion and Recommendations}
\label{ch:conclusion}

\textit{We summarize the contributions of the dissertation and
provide recommendations for further work.}

\section{Conclusion}

In this dissertation, we have proposed and evaluated a new method for
bootstrapping scene-specific anomalous human behavior detection
systems that incrementally learns behavior models without requiring
storage of large databases of training data. The method requires
minimal involvement of a human operator; the only required action is
to label the patterns in a small bootstrap set as normal or anomalous
and then to label false positive alarms as normal when they occur.

Our contributions are five-fold. First, we have developed a blob
extraction and appearance-based blob tracking method (explained in
Chapter~\ref{ch:blobanalysis}) that segments blobs and generate blobs'
trajectories as input to the behavior modeling module.

Second, we propose a new method for detecting shadows using a simple
maximum likelihood approach based on color information (explained in
Chapter~\ref{ch:shadow}). We extend the deterministic nonmodel-based
approach, designing a parametric statistical model-based approach.

Third, we propose and evaluate a new method for clustering human
behaviors (explained in Chapter~\ref{ch:clustering}). The method can
be used to bootstrap an anomaly detection module for intelligent video
surveillance systems. The combination of DTW with the type of linear
HMMs we use in this work is extremely effective in separating
anomalous from typical behaviors on real-world testbed video
surveillance data.

Fourth, we propose a \textit{semi-supervised} method for automatic
identification of suspicious behavior from a small bootstrap set
(explained in Chapter~\ref{ch:batch}). The method partitions the
bootstrap set into clusters then assigns new observation sequences to
clusters based on statistical tests of HMM log likelihood scores.

Fifth, we propose an incremental behavior modeling and suspicious
activity detection method (explained in Chapter~\ref{ch:incremental})
that incrementally learns scene-specific statistical models of human
behavior without requiring storage of large databases of training
data.

The experimental results presented in this dissertation are extremely
promising, demonstrating that our approach is a practical and
effective solution to the problem of inducing scene-specific
statistical models useful for bringing suspicious behavior to the
attention of human security personnel. Deploying our system on a large
video sensor network would potentially lead to substantial increases
in the productivity and proactivity of human monitors.

Next, I provide recommendations for future research.

\section{Recommendations}

Our approach can be extended to larger-scale situations as long as
typical behavior can be modeled in terms of the spatio-temporal motion
of foreground blobs.  It could also be extended to recognize more
complex events involving multiple persons --- we believe that we can
handle interactions between pedestrians, for example in pickpocket and
assault events, within the current framework of temporal statistical
models for individual humans by including observation features that
characterize a person's interaction with others while in the scene.
Integrating with pedestrian detection and tracking methods that rely
on body part detection and tracking rather than motion blob tracking
might help capture a larger proportion of the kinds of unusual
behaviors we would observe in building entrances or office hallways.

There are a few limitations to our current method.  Although, it would
work well for most building entrances, office building hallways, and
similar environments, the blob tracking process would not be robust
for scenes with dense crowds. Integrating with pedestrian tracking
methods for crowds \shortcite{ali12crowds} would be a potential
solution to the problem.

In some cases, our shadow detection method misdetects shadow pixels as
object pixels due to similar colors between the object and the
background and unclear background texture in shadow
regions. Incorporating geometric or shadow region shape priors would
potentially improve detection and discrimination rates.

Next, constructing a fixed codebook to quantize the feature space may
be inappropriate for incremental approaches since the codebook would
need to be revised over time to account for changing ``typical''
behaviors.  It may be better to take a probabilistic generative
approach to the assignment of feature vectors to discrete categories
rather than making hard assignments.

Also, our current system can add new HMMs but cannot remove them. Over
time, the number of HMMs would grow without bound.  It would be better
to periodically merge similar models or remove old models that no
longer represent typical behavior. Since each HMM has the same
structure, it would be very straightforward to check for pairs of
similar models and merge them.

The clustering results suggest it is likely that the patterns DTW
groups together are perfectly suited for modeling by linear HMMs. We
plan to further explore this idea in future work.

In the experimental results for our incremental method, we have shown
that local event processing is more effective than global event
processing on our data set.  It would be interesting to compare to a
hybrid approach in which we apply the global method to blocks in an
image rather than the whole image.  

Lastly, we will explore integrating the behavior understanding and
anomaly detection algorithms into a complete video surveillance system
such as ZoneMinder \shortcite{zoneminder}.

\FloatBarrier



% --- Maybe use later ---

%Our method relies on a multiple blob tracking approach that maintains
%individual identity through blob merges and splits.

%In future work, we plan to address these limitations.  

%In future work, we plan to address these
%limitations.
