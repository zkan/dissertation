\documentclass{article}

\usepackage{a4}
\usepackage{url}
\usepackage{amsmath}
\usepackage{setspace}
\usepackage{times}

\newcommand{\authnote}[1]{\begin{center}\fbox{\begin{minipage}{6.3in}
{#1}\end{minipage}}\end{center}}
\renewcommand{\vec}[1]{\mathbf{#1}}

\textwidth 6in
\oddsidemargin 0.15in
\topmargin -0.5in
\textheight 10in
\doublespacing

\date{}

\begin{document}

\pagestyle{empty}
\thispagestyle{empty}

%\begin{center}
%{\LARGE \bf \noindent Incremental Behavior Modeling and Suspicious
%  Activity Detection}
%\end{center}

\begin{table}
    \begin{tabular}{ll}
        {\large \bf Name:}  & {\large Kan Ouivirach} \\
        {\large \bf ID:}    & {\large 107830} \\
        {\large \bf Title:} & {\large Incremental Behavior Modeling 
                              and Suspicious Activity Detection} \\
        {\large \bf Type:}  & {\large Dissertation}
    \end{tabular}
\end{table}

\vspace{0.3in}
{\bf \noindent Abstract}\\

\noindent Video surveillance systems have become widespread tools for monitoring
and law enforcement in public areas.  Due to increasing crime, video
surveillance systems are being deployed in more and more places.
Video surveillance systems are needed to help security personnel
prevent and respond to criminal activity in a timely fashion.
However, human monitoring becomes increasingly expensive and
ineffective as the amount of video data increases. Also, manual review of
all video footage is time-intensive and error-prone.\\

\noindent Automated anomaly detection can help to improve the effectiveness of
human observers by separating the video stream into a sequence of
``normal'' and ``unusual'' events. However, much of the existing work
has many limitations in this direction. Oftentimes, separate models
for each distinct a priori known class of ``normal'' behavior are
assumed.  This could lead to the problem of typical behavior evolving
and becoming more diverse to the point that the false alarm rates
increase.  The naive solution would be to retain all of the
observation data and retrain the system periodically, but this
requires storing all of the incoming data, requiring too much disk
space. In this dissertation, we therefore propose and evaluate an
efficient method for incremental automatic identification of
suspicious behavior in video surveillance data.\\

\noindent First, we develop a blob extraction method that segments blobs and an
appearance-based blob tracking method that uses a forward-backward
overlap method and color coherence vectors (CCVs) to maintain identity
through blob merging and splitting cases.\\

\noindent Second, we propose a new method for detecting shadows using a simple
maximum likelihood formulation based on HSV color information. We find
that the method outperforms standard shadow detection methods on three
different real-world video surveillance data sets.\\

\noindent Third, we propose a new method for clustering human behaviors in the
context of video surveillance that is suitable for bootstrapping an
anomaly detection module for intelligent video surveillance
systems. We show that the method is extremely effective in separating
anomalous from typical behaviors on real-world testbed video
surveillance data.\\

\newpage

\noindent Fourth, we propose a \textit{semi-supervised} batch anomaly
detection method that self-calibrates itself from a small bootstrap
set in which each bootstrap sequence is manually labeled as normal or
suspicious by a human operator. Our method proves extremely effective,
with a very low false alarm rate at a 100\% hit rate.\\

\noindent Finally, we propose an effective behavior modeling and suspicious
activity detection method extending the batch method that
incrementally learns scene-specific statistical models of human
behavior without requiring storage of historical data. The incremental
method's false alarm rate drops below that of the batch method on the
same data.\\

\noindent In experimental evaluations on real-world testbed video surveillance
data sets, the proposed methods prove to be practical and effective at
inducing scene-specific statistical models useful for bringing
suspicious behavior to the attention of human security personnel.\\

\noindent {\bf Keywords:} hidden Markov models, incremental learning, 
behavior clustering, sufficient statistics, anomaly detection, bootstrapping,  
multiple blob tracking, shadow detection

\end{document}

