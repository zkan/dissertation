\documentclass{article}

\usepackage{a4}
\usepackage{url}
%\urlstyle{rm}  % or sf, depending on the font
\usepackage{amsmath}
%\usepackage [table]{xcolor}
%\usepackage{apacite}

\newcommand{\authnote}[1]{\begin{center}\fbox{\begin{minipage}{6.3in}
{#1}\end{minipage}}\end{center}}
\renewcommand{\vec}[1]{\mathbf{#1}}

\textwidth      6.75in
\oddsidemargin -0.25in
\topmargin     -0.75in
\textheight     10in

\title{\textbf{Response to External Examiner Report}\\[0.5em] 
    {\small ``Incremental Behavior 
    Modeling and Suspicious Activity Detection''}}

    \author{Kan Ouivirach}
\date{June 11, 2013}

\begin{document}

\maketitle

\noindent
I would like to thank Prof.\ James Clark for
the careful review and constructive comments. I have
carefully studied the comments, addressed them, and incorporated
corresponding changes to my dissertation. \\ 

%\noindent 
%I summarize the major changes to the manuscript as follows.
%content in each section. The major changes include:

%\begin{itemize}
%    \item adding a new Related Work section to further extend the survey
%        of related topics in the literature;
%    \item dropping the comparison to the traditional machine learning
%        methods since the results are already published in our conference
%        paper [41];
%    \item adding the comparison with HMM-based methods using a global
%        representation and scoring methods similar to those of Xiang and
%        Gong [33].
%\end{itemize} 

%\noindent
%\textbf{References:}\\

%\noindent
%[33] T. Xiang, S. Gong, Incremental and adaptive abnormal behaviour
%detection, Computer Vision and Image Understanding (CVIU) 111 (1)
%(2008) 59--73. \\

%\noindent
%[41] K. Ouivirach, S. Gharti, M. N. Dailey, Automatic suspicious
%behavior detection from a small bootstrap set, in: VISAPP, 2012,
%pp. 655--658.\\

\noindent Please see my response to the comments below.

\vspace{0.5cm}

\textit{I found this thesis to be well written and easy to read. I found the 
    technical work rather pedestrian (no pun intended), however. As a development 
    of a well-performing anomaly detection system for single moving objects, 
    the thesis work is acceptable, but it does not provide much in the way of 
    an advance in the state-of-the art. The thesis does a fair job of relating 
    previous work in areas directly related to the proposed approaches, but omits 
    mention of current state-of-the-art research in the area of human activity 
    recognition and detection. Finally, the thesis is deficient in the experimental 
    validation of the proposed approaches. Only a single, rather simple, experimental 
    situation was considered. The generality and robustness of the proposed approach 
    cannot be judged with such a limited range of test scenarios. The important 
    case of multiple objects was not at all considered, not even at a hypothetical 
    level. A discussion of the failure modes of the proposed approach was not provided. 
    This is extremely important as it suggests how the method can be improved, and 
    for what range of applications is appropriate. The experimental scenarios 
    should accordingly include those in which the system will fail.} \\

\textit{The thesis was rather short as Ph.D.\ theses go, and it appears that this 
    is due to a lack of detail in the description of the various algorithms, 
    the aforementioned sparsity of the experimental scenarios, and a lack of 
    informed and insightful discussion of the implications of the experimental 
    results. The discussion sections of each chapter are basically just brief 
    summaries of the results.} \\ 

\textit{The following are some detailed comments regarding the thesis text.} \\

\textit{The statement of the three main challenges for human behavior 
    understanding, on page 2, is too strong. Point 1 states that ``It is 
    impossible to store all data in a system.'' It is not impossible, 
    although it may be impractical. Point 2 states that ``Real human behavior 
    is always ambiguous;.'' This is not the case, human behavior is sometimes 
    unambiguous. Point 3 states that ``Unusual behavior is rare and diverse; 
    therefore we cannot learn a model for it.'' This is not true either, 
    since we can always learn a model, it just might not be a very good model. 
    Also, it is possible, but probably impractical, to acquire sufficient data 
    for a good model even for infrequent events.}

\authnote{I am thankful for the observations and corrections. I have
    revised the statements to be less dogmatic and more accurate.}

\newpage

\textit{There should be a more in-depth review of the state-of-the-art in 
    surveillance systems. By restricting the review to commercial and open-source 
    systems the thesis misses out on the most advanced techniques. Commercial and 
    open-source systems always lag behind the state-of-the-art by many years. 
    For examples of such, see the special issue of Machine Vision and Applications, 
    August 2007, Volume 18, Issue 3--4, on ``Novel concepts and challenges for the 
    next generation of video surveillance systems.'' There should also be a review 
    of recent work in human activity analysis, a currently hot area in machine vision 
    research (e.g.\ see the paper Turaga, P.; Chellappa, R.; Subrahmanian, V. S.; 
    Udrea, O., ``Machine Recognition of Human Activities: A Survey,'' Circuits and 
    Systems for Video Technology, IEEE Transactions on, vol.18, no.11, pp.1473--1488, 
    Nov.\ 2008). An example of recent high-quality work in this area is Weilong Yang; 
    Yang Wang; Mori, G., ``Recognizing human actions from still images with latent poses,'' 
    Computer Vision and Pattern Recognition (CVPR), 2010 IEEE Conference on, vol., no., 
    pp.2030--2037, 13--18 June 2010.}

\authnote{I have further extended the literature in Chapter 1 as suggested.}

\textit{In section 2.3 a simple changed pixel measure is used to detect motion in 
    an image. No details are given, however, such as the choice of the intensity 
    difference threshold value. Such an approach can fail in common surveillance 
    applications where the camera is subject to small vibration. These vibrations 
    cause small shifts of the entire image, which can be sufficient to make a 
    large number of pixels change their values significantly.}

\authnote{This is indeed a limitation of our system.  I have added the following text 
    to the motion detection section: ``Through some preliminary empirical experimentation 
    I found that 300 was a good threshold. While this simple approach worked well in 
    our experimental setup, it would not work as well in a situation where the camera 
    is subject to vibration, causing a large number of pixels to change values 
    significantly.''}

\textit{On page 19 the formulae for the $dx$ and $dy$ should be given. If these 
    are normalized (since they are referred to as a unit vector) then it is 
    improper to label them as $dx$ and $dy$ (since that implies an incremental 
    vector, not a unit vector). Also, if they define a unit vector, then it 
    is redundant to use both of them in the feature vector. You should instead use 
    a single angle parameter.}

\authnote{While angle plus velocity would indeed be a minimal representation, we chose 
    $x$, $y$, and velocity because of the small practical
    problem of calculating 
    Euclidean distances for angles. For instance, the distance between 1 degree
    and 359 degrees is 2, 
    not 358. Addressing this would require modifying the kmeans
    library's source code.  While feasible, I did not think this would
    change the result significantly.
    To address the point of naming, I have
    renamed $dx$ to $\theta_x$ and $dy$ to $\theta_y$ to 
    make the text more clear.}

\textit{It is important to say how you improved upon the method of Gharti (2010) 
    for blob-tracking. The reader should know exactly what your contribution was. 
    Just saying that you improved the code is not enough. You have to say what 
    exactly you did.}

\authnote{Actually, Gharti and I worked together on the first version of
the blob tracking library.  The result became part of her
master's thesis.  After she
completed the thesis, I did some additional work to adapt the library
for my purposes.  I have revised the section to make this more clear.}

\textit{You say that you did not use Kalman filters or other approaches (presumably 
    particle-filter methods) in the blob tracking since your data has little 
    noise and the results are being coarsely quantized. However, the difficulties 
    in tracking that are handled by particle-filter type methods are not those 
    caused by noise, but those caused by multiple moving objects, with the attendant 
    occlusions and ambiguity. Presumably these issues are present in your application 
    as well (as you later note in section 2.7).}

\authnote{This is an artifact of a response to a reviewer's 
    specific comment about Kalman filtering the blob position on our Pattern Recognition 
    manuscript and we really only meant the text to relate to that idea. 
    Tracking blobs with particle filters before quantization could indeed improve 
    the tracking results.  We have revised the text regarding Kalman filters to 
    be more specific to the blob position from the tracker.}

\newpage

\textit{You should define all acronyms used in the thesis at their first appearance 
    in the text (for example, HMM should be defined at the start of section 2.6). 
    It is good practice to include a table of acronyms and symbols in the front matter 
    of the thesis.}

\authnote{I have fixed the mistake and now as suggested, I
    include a table of acronyms and symbols in the front matter.}

\textit{On page 22 it is stated that prior to clustering you normalize each feature by 
    a $z$-score. What is a $z$-score in this context? Is it based on a statistical 
    model of the feature? Details should be given here, at least provide an equation 
    showing how the $z$-score is computed and the normalization done.}

\authnote{I have added details as suggested.}

\textit{The way in which equation 3.1 is written makes it unclear that (as I assume) 
    the three likelihoods $P(H_\text{diff})$, $P(S_\text{diff})$, and $P(V_\text{diff})$ 
    are multiplied together. Perhaps put in a multiplication symbol ($*$ or $\times$). Also, 
    it is improper to call this the ``measurement likelihood,'' as it is actually the 
    probability of measurement given the classification A, whereas it is the likelihood 
    of the classification given the measurement, i.e.\ $L(A_{xy}=\text{sh} \mid M_{xy}) = 
    P(M_{xy} \mid A_{xy}=\text{sh})$.}

\authnote{I am thankful for the careful review. I have fixed the text as 
    suggested.}

\textit{How are the Gaussian model parameters estimated (in section 3.2.1)? The 
    distributions for the true object pixels are clearly non-Gaussian so there 
    are many ways that a Gaussian could be fit to these. What approach did you use? 
    Did you just use the sample means and variances? A minimum variance estimator 
    might be better.}

\authnote{It is true that the distrubutions are very non-Gaussian. I
  plan to extend this chapter in future work, but in this version, I
  stuck to a relatively simple approach, so yes, I simply use the
  sample mean and variance.  However, on extension of the work, we
  will definitely investigate better estimators and different
  distributions.  I have added this explanation to Section 3.2.1.}

\textit{In section 4.2.2 it is said that morphological operations are used. 
    More detail should be given for these. In general, enough detail should 
    be given so as to allow someone to fully implement the proposed approach 
    and replicate the experiments done.}

\authnote{I have added details to the section as suggested.}

\textit{Section 4.2.2 states that the evaluation is limited to a single moving blob. 
    It is said that this is for simplicity, but it also prevents the approach from 
    being evaluated on the more usual and interesting problem of multiple targets. 
    Saying that the restriction is for simplicity implies that it is not imposed 
    due to the inability of the proposed approach to handle the multiple target 
    case. If the proposed approach does in fact work for multiple targets this 
    should be stated and verified, as it would be a strong advantage of the approach. 
    If the approach does not work in this case, then it is misleading to say that 
    the single blob case was used for simplicity, when in fact it is being imposed 
    due to the approach failing. In either case the proposed approach should be 
    tested on multiple targets to see how it performs and where it might fail.}

\authnote{I regret that it is unclear in the description of our methodology. 
    Only the evalution in Chapter 4 is limited to a single moving blob. 
    We extended the experiments to multiple moving blobs in later chapters. 
    I have revised the description to eliminate the confusion.}

\textit{How is the empirical ``tuning'' described in section 4.2.2 actually done? 
    Details are needed!}

\authnote{We set up our system and observed the scene for a few days
  before we started collecting data. We manually fine-tuned the frame
  buffer length and motion detection thresholds for triggering events
  to achieve good results per manual observation. Other parameters
  critical to the modeling such as the number of $k$-mean clusters
  were set through cross validation.  I have modified the section to
  make this more clear.}

\newpage

\textit{The equation on page 36 (which should have an equation number!) 
    defines a (log) likelihood, but the text incorrectly refers to this 
    as the likelihood of the sequence $O_i$. This indicates a misunderstanding 
    of the meaning of the term ``likelihood.'' The quantity $L_i$ is the log 
    of the probability of the sequence $O_i$ given the HMM $M_c$, and is also 
    the log of the likelihood of the HMM $M_c$ given the sequence $O_i$. 
    Note the difference!}

\authnote{I regret the confusion caused by our description and have fixed it.}

\textit{Figures 4.4, 5.2 and 6.1 are duplicates of each other. Only one need 
    be given. These figures (or figure) should show one or more examples of 
    the anomalous behaviors. The text should also describe the nature of 
    typical anomalous behaviors that were observed (e.g.\ a person standing 
    still or dancing, or stopping halfway and turning back in the direction 
    they came).}

\authnote{I have added one more example of common behaviors in Figure 5.2 
    to distinguish the data in Chapter 4 and added more typical abnormal 
    behaviors in Figure 5.5. I have also revised the text as suggested.}
    
\textit{In table 4.1 it is evident that the clusters do not separate the 
    Walking In and Cycling In behaviors. This implies that the HMMs are 
    not discriminative enough and perhaps could benefit from additional 
    training examples. Of course, for the purposes of deciding on usual 
    versus anomalous this is acceptable, but that leads to the conclusion 
    that a simpler training of HMMs would work just as well.}

\authnote{This is a good point. Since the current system treats each
  ($z$-scaled) feature equally when it performs blob feature vector
  discretization, important fine distinctions for some features can be
  lost.  Adding more weight to features such as blob size or aspect
  ratio could help distinguish these behaviors.  I have modified the
  text to include this point. \\

  Regarding a simpler HMM training method, we could have 
  human operator select the groups of similar behaviors, train one model 
  on each group, and try to come up with a set of thresholds for 
  typical behavior discrimination and anomaly detection. However, based on my 
  previous empirical experience in my Master's thesis [1], a simpler method 
  would not work. The problem is that the appropriate threshold may vary 
  within one scene from model to model. Threfore, the threshold selection 
  with the simpler method becomes painful. \\
  
  {\bf Reference:} 

  [1] Ouivirach, K.\ (2008).\ \textit{Human behavior profiling for a video 
  surveillance system}. Unpublished Master's thesis, Computer Science and 
  Information Management (CSIM), Asian Institute of Technology (AIT).}

\textit{It is misleading to claim that the separation of anomalous and 
    typical behaviors in Table 1 is 100\%. It is not, since there are 
    still unseparated behaviors in the single sequence clusters. A manual 
    classification of these single-sequence clusters could result in 
    the anomalous behaviors being classified as one of the typical behaviors.}

\authnote{It is true that it would be difficult to separate these
  single-sequence clusters. I have revised the claim accordingly.}

\textit{The conclusion stated in section 4.3.3 that ``our method achieves 
    perfect separation of anomalous and typical behaviors'' is completely 
    unwarranted. All that you have done is demonstrated this perfect separation 
    in a single restricted dataset. You would need to repeat this experiment 
    over a very much larger database of examples before anyone would take 
    this conclusion seriously. The limited experimentation is a major weakness 
    of the thesis. Nothing can really be concluded from it.}

\authnote{I thought it obvious that the conclusion would be taken as
  specific to the restricted dataset used in Chapter 4 (please see the
  response above regarding the single-blob limitation that is
  eliminated in later chapters), but I have modified the claim to make
  it more clear.\\

  The method requires at least a week of video data. Other data sets I
  found, such as the PETS data set, which mainly focuses on tracking
  eveluation, or the ViSOR data set, contain videos recorded less than 
  one week for each camera. Therefore, they are inappropriate
  for our method. \\
  
  Unfortunately, so far, we have not found any large
  data set that is appropriate yet.  Since no existing dataset is
  suitable, we created our own. It is freely available for others to
  experiment with at
  \url{http://se.cs.ait.ac.th/~kan/Datasets/CSIM-Forecourt-Videos/}.}

\textit{Is the $\theta_z$ of Algorithm 2 the same as the quantity $p_c$ 
    defined in section 5.2.1?}

\authnote{They are different. The $p_c$ is the optimal rejection threshold 
    for cluster $c$ whereas the $\theta_z$ is the alarm threshold.}

\textit{Is the testbed described in section 5.4 the same as that used in 
    chapter 4?}

\authnote{The testbed data in Chapter 4 and in Section 5.4 are different. 
    We evaluate the testbed data in Chapter 4 with only single moving blobs 
    and we extened this testbed data to include multiple moving blobs in 
    Section 5.4 or Chapter 5. Please see my response regarding this issue 
    above.}

\textit{The graphs in figure 5.3 should be enlarged. Perhaps array these as 2 
    columns by 3 rows, and fill an entire page. Otherwise it is difficult to 
    read the legend and labels of the graphs. The examination copy that I was 
    given did not have any color, so the captions should be altered to remove 
    any reference to color. Markers could be added to the lines on the graph 
    to help distinguish them. The axes need to be labeled!}

\authnote{I haved increased the size of figures, fixed the labels, and revised 
    the caption as suggested.}

\textit{Table 5.1 does not show ``perfect accuracy'' as stated in section 5.4.1, 
    since, as in chapter 4, the one-sequence clusters do not separate the 
    anomalous from typical behaviors. This has to be done manually, and it 
    is not demonstrated that manual separation can be done perfectly.}

\authnote{As above, it is true that it would be difficult to separate
  the single-sequence clusters.  I have revised the section
  accordingly.}

\textit{The version of the thesis I was given did not have color, so the different 
    curves in figure 5.4 could not be distinguished. These should have different 
    line types, or have distinguishing markers added.}

\authnote{I have fixed it.}

\textit{On page 50 it is stated that the ROC (figure 5.4) reveals that a 
    threshold of -3.259 achieves zero false negatives. However, there are 
    no indications of the threshold values on figure 5.4 so I do not see 
    how this threshold value of -3.259 can be obtained by looking at 
    figure 5.4.}

\authnote{I have fixed it.}

\textit{The acronyms (e.g.\ TP, FP, TN, etc.) used in Table 5.2 should be 
    defined in the table caption.}

\authnote{I now define the acronyms in the table captions throughout the 
    dissertation. I also put a list of acronyms in the front matter.}

\textit{How do the results of chapter 5 vary with changes in the size of the 
    bootstrap set? Is there an optimal size?}

\authnote{I have performed some brief experimentation to determine how
  the size of bootstrap set affects the results. I found that with
  more bootstrap data we tend to discover more compact behavior
  classes, leading to better anomaly detection rates in practice. \\
    
  However, since a larger bootstrap set will generally
  contain more variable
  behavior, our approach consequently creates more
  models to handle the more variable set of patterns.
  The effect is that we require more interaction
  with the operator determine whether each behavior cluster is normal or
  abnormal.\\
  
  In this sense, the
  ``optimal'' bootstrap set size would thus be a size sufficient to
  include most typical behaviors (1--2 days) but no more.\\

  I have added a note to this effect to Section 5.4.2.1.
  }

\authnote{I thank Prof.\ Clark again for the careful review that has
  helped me to substantially improve my dissertation.}

\end{document}

