
%--------------------------------------------------------------------
\section{External Examiner's Comments}
%--------------------------------------------------------------------

\begin{frame}
    \frametitle{External Examiner's Comments}
    \framesubtitle{External Examiner}

    My external examiner (EE):

    \medskip

    Prof.\ James J.\ Clark \\
    Dept.\ of Electrical and Computer Engineering \\
    McGill University, Canada

\end{frame}

%--------------------------------------------------------------------

\begin{frame}
    \frametitle{External Examiner's Comments}
    \framesubtitle{Comments}

    \begin{itemize}
        \item Well written and easy to read.
        \item Well-performing anomaly detection system for single 
            moving objects, the work is acceptable, but it does not 
            provide much in the way of an advanced in the state-of-the-art. 
        \item The thesis does a fair job of relating previous work in area 
            directly related to the proposed methods, but omits the mention 
            of currect state-of-the-art research in the area of human 
            activity recognition and detection.
        \item Only a single, rather simple, experimental situation was 
            considered. The generality and robustness of the proposed 
            method cannot be judged with such a limited range of test 
            scenarios. 
    \end{itemize}

\end{frame}

%--------------------------------------------------------------------

\begin{frame}
    \frametitle{External Examiner's Comments}
    \framesubtitle{Comments (cont.)}

    \begin{itemize}
        \item Section 4.2.2 states that the evaluation is limited 
            to a single moving blob. Saying that the restriction 
            if for simplicity implies that it is not imposed due to 
            the inability of the proposed approach to handle the multiple 
            target case.
        \item It is evident that the clusters do not separate the Walking In 
            and Cycling In behaviors. This implies that the HMMs are not 
            discriminative enough and perhaps could benefit from additional 
            training examples.
        \item Conclusion in Section 4.3.3 states that ``out method achieves 
            perfect separation of anomalous and typical beahviors'' is completely 
            unwarranted. All that you have done is demonstrated this perfect 
            separation in a single restricted dataset.
    \end{itemize}

\end{frame}



