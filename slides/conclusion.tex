
%--------------------------------------------------------------------
\section{Conclusion and Recommendations}
%--------------------------------------------------------------------

\begin{frame}
    \frametitle{Conclusion and Recommendations}
    \framesubtitle{Conclusion}

    Our contributios are five-fold.
    \begin{enumerate}
        \item Develop a blob extraction and appearance-based 
            blob tracking method that segments blobs and generate 
            blobs' trajectories as input to the behavior modeling module.
        \item Propose a new method for detecting shadows using a 
            simple maximum likelihood approach based on color information. 
            We extend the deterministic nonmodel-based approach, designing 
            a parametric statistical model-based approach.
        \item Propose and evaluate a new method for clustering human 
            behaviors. This method can be used to bootstrap an anomaly 
            detection module for intelligent video surveillance systems. 
        \item Propose a \textit{semi-supervised} method for automatic 
            identification of suspicious behavior from a small bootstrap 
            set.
    \end{enumerate}

\end{frame}

%--------------------------------------------------------------------

\begin{frame}
    \frametitle{Conclusion and Recommendations}
    \framesubtitle{Conclusion (cont.)}

    \begin{enumerate}
        \setcounter{enumi}{5}
        \item Propose an incremental behavior modeling and suspicious 
            activity detection method that incrementally learns scene-specific 
            statistical models of human behavior without requiring storage of 
            large databases of training data. 
    \end{enumerate}

    \medskip

    The experimental results are extremely promising, demonstrating that our 
    approach is a practical and effective solution to the problem of inducing 
    scene-specific statistical models useful for bringing suspicious behavior 
    to the attention of human security personnel.

\end{frame}

%--------------------------------------------------------------------

\begin{frame}
    \frametitle{Conclusion and Recommendations}
    \framesubtitle{Recommendations}

    \begin{enumerate}
        \item Our approach can be extended to larger-scale situations 
            as long as typical behavior can be modeled in terms of the 
            spatio-temporal motion of foreground blobs. 
        \item It could also be extended to recognize more complex events involving 
            multiple persons -- we believe that we can handle interactions between 
            pedestrians within the current framework of temporal statistical 
            models for individual humans by including 
            observation features that characterize a person's interaction with others 
            while in the scene. 
        \item Integrating with pedestrian detection and tracking 
            methods that rely on body part detection and tracking rather 
            than motion blob tracking might help 
            capture a larger proportion of the kinds of unusual 
            behaviors we would observe in building entrances or office hallways.
    \end{enumerate}

\end{frame}

%--------------------------------------------------------------------

\begin{frame}
    \frametitle{Conclusion and Recommendations}
    \framesubtitle{Recommendations (cont.)}

    \begin{enumerate}
        \setcounter{enumi}{3}
        \item Although, it would work well for most building entrances, 
            office building 
            hallways, and similar environments, the blob tracking 
            process would not be
            robust for scenes with dense crowds. Integrating with 
            pedestrian tracking 
            methods for crowds (Ali \& Dailey, 2012) would be a 
            potential solution 
            to the problem.
        \item In some cases, our shadow detection method misdetects 
            shadow pixels 
            as object pixels due to similar
            colors between the object and the background and unclear 
            background 
            texture in shadow regions. Incorporating geometric or shadow 
            region 
            shape priors would potentially improve detection and 
            discrimination rates. 
    \end{enumerate}

\end{frame}

%--------------------------------------------------------------------

\begin{frame}
    \frametitle{Conclusion and Recommendations}
    \framesubtitle{Recommendations (cont.)}
        
    \begin{enumerate}
        \setcounter{enumi}{5}
        \item Next, constructing a fixed codebook to quantize the 
            feature space 
            may be inappropriate for incremental approaches since 
            the codebook would 
            need to be revised over time to account for changing 
            ``typical'' behaviors. 
            It may be better to take a probabilistic generative approach 
            to the assignment of 
            feature vectors to discrete categories rather than making 
            hard assignments.
        \item Also, our current system can add new HMMs but cannot remove them. 
            Over time, the number of
            HMMs would grow without bound. It would be better to periodically merge 
            similar models or remove
            old models that no longer represent typical behavior. Since each HMM has 
            the same structure, it
            would be very straightforward to check for pairs of similar models 
            and merge them.
    \end{enumerate}

\end{frame}

%--------------------------------------------------------------------

\begin{frame}
    \frametitle{Conclusion and Recommendations}
    \framesubtitle{Recommendations (cont.)}
        
    \begin{enumerate}
        \setcounter{enumi}{7}
            \item The clustering results suggest it is likely that the patterns DTW 
                groups together 
                are perfectly suited
                for modeling by linear HMMs. We plan to further explore this idea 
                in future work.
            \item In the experimental results for our incremental method, we have 
                shown that local 
                event processing is more effective than global event processing 
                on our data set. 
                It would be interesting to compare to a hybrid approach in which we 
                apply the global 
                method to blocks in an image rather than the whole image.
            \item Lastly, we will explore integrating the behavior understanding and 
                anomaly detection algorithms 
                into a complete video surveillance system such as 
                ZoneMinder (Coombes, 2007).
    \end{enumerate}

\end{frame}

